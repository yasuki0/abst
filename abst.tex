\documentclass[a4j,twocolumn]{jsarticle}

\usepackage[dvipdfmx]{graphicx}

\setlength{\textheight}{275mm}
\headheight 5mm
\topmargin -30mm
\textwidth 185mm
\oddsidemargin -15mm
\evensidemargin -15mm
\pagestyle{empty}

\begin{document}
\title{Mg-LPSOのL12 クラスターの第一原理計算}
\author{情報科学科 西谷研究室 3539 山本 泰基}
\date{}
\maketitle
\section{背景}
我々はLPSO 構造の生成機構について,「積層欠陥部に L12 クラスターが形成され,そこから排斥された Zn, Y が,濃化して新たな L12 クラスターを形成する」というシナリオを立てた [1]. これまでの研究では,L12 クラスターから 1 層ずつ離れた位置に,孤立した溶質原子あるいは Zn-Yペアを挿入して第一原理計算をおこない,系全体のエネルギーを比較した. 第一原理計算とは,量子力学を支配するシュレディンガー方程式を正確に解いて, 原子の種類だけから電子構造を求め, いろいろな物性を予測する計算である. 計算の結果, 系全体のエネルギーは溶質原子と L12 クラスターとの距離が離れるにつれ単調に減少し安定となった.しかし, それは中周期的に溶質原子が濃化するという予想に反する結果であった. 本研究では,これまで考慮してきたより大きなサイズの溶質原子のクラスター集団を仮定して同様の計算をおこなう.

\section{手法}
清原らは,L12 クラスターを hcp 構造に強引に導入すると,2 つに分裂した small cluster が生成すると予測している [2].このサイズは実験的には奥田らが報告しているクラスターサイズに近い [3].この small cluster を slab モデルに挿入して, VASP を用いて第一原理計算を行い,構造緩和したエネルギーを求める. 本研究で行う構造緩和は,内部・外部緩和両方を用いて,最安定構造を求め, エネルギー値を算出する.

\section{考察}
まず,L12 クラスターがどのように分離した時に最も安定となるかを,small cluster の生成エネルギーを比較することで確かめた.幾何学的な可能性として考えやすい上下および左右 に分割した.その結果,上下に分割した際の small cluster の生成エネルギーが低く,安定構造となった.上下に分割した small cluster を,積層欠陥にある L12 クラスターから離れた位置へ図 1 に示したように挿入した. 第一原理計算によって得られた系全体のエネルギーを図 2 に示した. 4 層離れた位置での計算についてはエネルギー値が収束していないが,他の層の計算結果から 距離が離れるに連れ単調減少を示すだけでなく,僅かではあるが中距離に安定位置がある傾向を示している. 今後は4層離れた位置での計算が収束に向かうよう分析し, 研究を継続する.


\section{参考文献}
[1] Y. Sakamoto, C. Shirayama, Y. Yamamoto, R. Kubo, M. Kiyohara, and S. R. Nishitani: Mater.Trans., 56(2015), 933.
\newline ~~~[2] M. Kiyohara, Y. Sakamoto, T. Yoshioka, S. Morishita, and S. R. Nishitani: proceedings of PRICM, (Kyoto 2016), to be published.
\newline ~~~[3] H. Okuda, M. Yamasaki, Y. Kawamura, M. Tabuchi, H. Kimizuka: Scientific Reports 5 (2015), 14186.
\end{document}