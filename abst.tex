\documentclass[a4j,twocolumn]{jsarticle}

\usepackage[dvipdfmx]{graphicx}

\setlength{\textheight}{275mm}
\headheight 5mm
\topmargin -30mm
\textwidth 185mm
\oddsidemargin -15mm
\evensidemargin -15mm
\pagestyle{empty}

\begin{document}
\title{Mg-LPSOのL1$_2$クラスターの第一原理計算}
\author{情報科学科 西谷研究室 3539 山本 泰基}
\date{}
\maketitle
\section{背景}
LPSO構造をもったMgは比降伏強度でジュラルミンを上回る特性を持ち,
かつ難燃性であるため次世代の航空機の構造材料として実用化が始まっている.
西谷研究室では,このLPSO構造の生成機構として「積層欠陥部に L1$_2$クラスターが形成され,そこから排斥された Zn, Yが,濃化して新たなL1$_2$クラスターを形成する」というシナリオを立て\cite{sakamoto},
第一原理計算を用いて系のエネルギーからそのシナリオの実現性を評価してきた. 
第一原理計算は,量子力学を支配するシュレディンガー方程式を精確に解いて, 原子の種類だけから電子構造を求め, いろいろな物性を予測する計算である. 
計算の結果, 系全体のエネルギーは溶質原子とL1$_2$クラスターとの距離が離れるにつれ単調に減少し安定となったが, それは中周期的に溶質原子が濃化するというLPSOの構造から予想される結果に反するものであった. 
本研究では,これまで考慮してきたサイズより大きなの溶質原子のクラスターを仮定して,同様の計算をおこなう.

六方最密充填ろっぽうさいみつじゅうてん

\section{手法}
清原らは,L1$_2$クラスターを母相のMgがとるhcp構造に強引に導入すると,2つに分裂したより小さなclusterが生成すると予測している\cite{kiyohara}.
このサイズは小角散乱の実験から奥田らが報告しているクラスターサイズに近い\cite{okuda}.
このsmall clusterを[slabモデル]に挿入して, VASP を用いて第一原理計算を行い構造緩和したエネルギーを求める. 周期的境界条件
本研究で行う構造緩和は,[内部・外部緩和両方]を用いて,最安定構造を求め,エネルギー値を算出する.

\section{考察}
まず,L1$_2$ クラスターがどのように分離した時に最も安定となるかを,small cluster の生成エネルギーを比較することで確かめた.
幾何学的な可能性として考えやすい上下および左右に分割した. 「図として入れたら?」
その結果,上下に分割した際の small cluster の生成エネルギーが低く,安定構造となった.
上下に分割したsmall clusterを,積層欠陥にある L12 クラスターから離れた位置へ図 1 に示したように挿入した. 
第一原理計算によって得られた系全体のエネルギーを図 2 に示した. 4 層離れた位置での計算についてはエネルギー値が収束していないが,他の層の計算結果から 距離が離れるに連れ単調減少を示すだけでなく,僅かではあるが中距離に安定位置がある傾向を示している. 

今後
今後は4層離れた位置での計算が収束に向かうよう分析し, 研究を継続する.
より遠くの離れた積層まで入れる
他のconfigurationも試す.
そうすると,より安定なエネルギーが見つかっちゃうかも...それもまた人生.

\begin{thebibliography}{9}
\bibitem{sakamoto}Y. Sakamoto, C. Shirayama, Y. Yamamoto, R. Kubo, M. Kiyohara, and S. R. Nishitani: Mater.Trans., 56(2015), 933.
\bibitem{kiyohara} M. Kiyohara, Y. Sakamoto, T. Yoshioka, S. Morishita, and S. R. Nishitani: proceedings of PRICM, (Kyoto 2016), to be published.
\bibitem{okuda} H. Okuda, M. Yamasaki, Y. Kawamura, M. Tabuchi, H. Kimizuka: Scientific Reports 5 (2015), 14186.
\end{thebibliography}


\end{document}